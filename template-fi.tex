% --- Template for thesis / report with tktltiki2 class ---

\documentclass[finnish]{tktltiki2}

% tktltiki2 automatically loads babel, so you can simply
% give the language parameter (e.g. finnish, swedish, english, british) as
% a parameter for the class: \documentclass[finnish]{tktltiki2}.
% The information on title and abstract is generated automatically depending on
% the language, see below if you need to change any of these manually.
% 
% Class options:
% - grading                 -- Print labels for grading information on the front page.
% - disablelastpagecounter  -- Disables the automatic generation of page number information
%                              in the abstract. See also \numberofpagesinformation{} command below.
%
% The class also respects the following options of article class:
%   10pt, 11pt, 12pt, final, draft, oneside, twoside,
%   openright, openany, onecolumn, twocolumn, leqno, fleqn
%
% The default font size is 11pt. The paper size used is A4, other sizes are not supported.
%
% rubber: module pdftex

% --- General packages ---

\usepackage[utf8]{inputenc}
\usepackage{moreverb} 
\usepackage{url}
\usepackage{lmodern}
\usepackage{microtype}
\usepackage{amsfonts,amsmath,amssymb,amsthm,booktabs,color,enumitem,graphicx}
\usepackage[pdftex,hidelinks]{hyperref}
\usepackage{setspace}
\doublespacing

% Automatically set the PDF metadata fields
\makeatletter
\AtBeginDocument{\hypersetup{pdftitle = {\@title}, pdfauthor = {\@author}}}
\makeatother

% --- Language-related settings ---
%
% these should be modified according to your language

% babelbib for non-english bibliography using bibtex
\usepackage[fixlanguage]{babelbib}
\selectbiblanguage{finnish}

% add bibliography to the table of contents
\usepackage[nottoc]{tocbibind}
% tocbibind renames the bibliography, use the following to change it back
\settocbibname{Lähteet}

% --- Theorem environment definitions ---

\newtheorem{lau}{Lause}
\newtheorem{lem}[lau]{Lemma}
\newtheorem{kor}[lau]{Korollaari}

\theoremstyle{definition}
\newtheorem{maar}[lau]{Määritelmä}
\newtheorem{ong}{Ongelma}
\newtheorem{alg}[lau]{Algoritmi}
\newtheorem{esim}[lau]{Esimerkki}

\theoremstyle{remark}
\newtheorem*{huom}{Huomautus}


% --- tktltiki2 options ---
%
% The following commands define the information used to generate title and
% abstract pages. The following entries should be always specified:

\title{Tietokantakyselyjen optimointi relaatiotietokannassa}
\author{Olli Rissanen}
\date{\today}
\level{Kandidaatintutkielma}
\abstract{Tutkielmassa tutustutaan tietokantakyselyjen optimointiin relaatiotietokantojen hallintajärjestelmien osalta sekä optimoinnin vaikutukseen kyselyjen suorituskyvyssä. }

% The following can be used to specify keywords and classification of the paper:

\keywords{Information systems → Query optimization}
\classification{} % classification according to ACM Computing Classification System (http://www.acm.org/about/class/)
                  % This is probably mostly relevant for computer scientists

% If the automatic page number counting is not working as desired in your case,
% uncomment the following to manually set the number of pages displayed in the abstract page:
%
% \numberofpagesinformation{16 sivua + 10 sivua liitteissä}
\begin{document}

% --- Front matter ---

\maketitle
\makeabstract
\tableofcontents
\newpage


% --- Main matter ---
\section{Johdanto}
%Konteksti
Modernit järjestelmät lisäävät jatkuvasti tietokantojen työtaakkaa tiedon määrän kasvaessa. Jotta tiedosta saadaan mahdollisimman paljon irti, tarvitaan tiedon hallitsemiseen yhä tehokkaampia työkaluja. % Tiedon hallitsemisesta ja keräämisestä voi muuten koittua rasitteeksi joka ylittää tiedosta saatavat hyödyt. 
Tietokannan suorituskyky on tärkeää koko järjestelmän suorituskyvyn osalta, sillä tiedon lukeminen massamuistista on hidasta verrattuna rekistereiden tai välimuistin käyttöön. Optimoimalla tietokantakyselyjen suoritusta voidaan vaikuttaa suoritettujen operaatioiden määrään sekä muistialueen kokoon ja siten vähentää tietokannan vasteaikaa sekä resurssien käyttöä. \cite{mor2012}
	
%Tietokannan suorituskyky on ydintekijä olio-relaatiokuvausta (ORM, object-relational mapping) käyttävissä ohjelmointikielissä. ORM luo relaatiotietokannan pohjalta käyttäjälle oliotietokannan, jonka suorituskyky on kuitenkin sidottu relaatiotietokantaan.
% 
%Erityisesti olio-relaatiokuvausta (ORM, object-relational mapping) toteuttavissa ohjelmointikielissä puhtaiden SQL-kyselyjen kirjoittaminen on siirretty käyttäjältä alemmalle tasolle.relaatiotietokantaan. Relaatiotietokannan kyselyjen optimoinnilla pystytään siten saavuttamaan .. [viite ois kiva]

Tietokantaa käytetään tietokannan hallintajärjestelmällä, joka on kokoelma ohjelmia tiedon tallentamiseen, muokkaamiseen, analysointiin ja keräämiseen tietokannasta. Hallintajärjestelmää käytetään kyselykielellä, joista esimerkiksi SQL \cite{5391051} on suunniteltu relaatiotietokantojen hallintajärjestelmille. Hallintajärjestelmän vastuulla on kyselyn muuttaminen tietokannan ymmärtämään muotoon säilyttäen kyselyn alkuperäisen tarkoituksen. Kyselyn optimointi on toteutettu automaattisena toimenpiteenä tietokannan hallintojärjestelmän sisältämässä kyselyn optimoijassa, ja kaikista hallintajärjestelmän komponenteista optimoijalla on suurin merkitys tietokannan suorituskykyyn. \cite{mor2012} Kyselyn optimoijan tavoitteena on minimoida itse optimointiin käytetty aika ja maksimoida optimoinnista saatu hyöty. \cite{jarke1984} 

Optimoija toimii etsien kyselyä vastaavat kyselysuunnitelmat ja valitsemalla niistä tehokkaimman. Kyselysuunnitelma sisältää sarjan algebrallisia operaatioita tietokannan relaatioille jotka tuottavat tulokseksi halutun vastauksen. Tietokantakyselyä vastaavia kyselysuunnitelmia voi olla useita, sillä kyselyjen algebralliset esitykset voidaan usein esittää monena loogisesti vastaavana esityksenä.  \cite{jarke1984} Algebrallista operaatiota kohden voi myös löytyä useita toteutuksia, kuten join-operaatiota toteuttavat merge join ja hash join. Saman kyselyn tuottamat kyselysuunnitelmat voivat olla suorituskyvyltään jopa eri suuruusluokassa. \cite{oracle2013refman}	
%Ongelma

Pohjan nykyaikaisille kyselyn optimoijille loi System R-hallintajärjestelmän \cite{astrahan1976system} sisältämä optimoija, jonka esittelivät Selinger et al. vuoden 1979 artikkelissaan Access Path Selection in a Relational Database Management System \cite{selinger1979access}. System R oli ensimmäinen relaatiomalliin \cite{codd1970relational} perustuva hallintajärjestelmä, ja moni nykyinen hallintajärjestelmä pohjautuu osittain System R:ään.

Optimointi on vaikea hakuongelma, jossa hakualue voi nousta erittäin suureksi. \cite{chaudhuri1998} Haasteeksi nousee kyselysuunnitelmien luominen ja niiden suorituskyvyn ennustaminen. Kaikkien mahdollisten kyselysuunnitelmien luominen on usein liian hidasta, joten optimoijan tulee valita pienin mahdollinen hakualue joka pitää sisällään halvimmat suunnitelmat. Suorituskyvyn ennustamisen ja hakualueen rajauksen lisäksi optimoija tarvitsee tehokkaan algoritmin koko hakualueen läpikäymiseen. On epärealistista odottaa kyselyn optimoijan aina löytävän parhaan kyselysuunnitelman, ja onkin tärkeämpää välttää huonoimpia suunnitelmia ja löytää hyvä suunnitelma. \cite{ramakrishnan2003database}

Kappale 2 sisältää esitiedot kyselyn optimoijan toiminnalle, jonka lisäksi kappaleessa perehdytään optimoijan rakenteeseen hallintajärjestelmän sisällä.  Kappale 3 käsittelee optimointiprosessin vaiheita yksityiskohtaisesti. Kappaleissa 4 ja 5 esitellään kyselyn optimoijan käyttöä MySQL-hallintajärjestelmällä.

%
%• Yleinen selostus tutkimusaiheen laajemmasta tutkimusalueesta, tutkimuksen kontekstista.
%• Ratkaistavan tutkimusongelman selostus (vaikeus, esteet, haasteet).
%• Lyhyt katsaus ongelman olemassa oleviin tai tavanomaisiin ratkaisuihin ja niiden rajoitteisiin.
%• Hahmotelma tai yhteenveto ehdotettavasta uudesta ratkaisusta.
%• Yhteenveto ratkaisun arvioimistavasta ja arvioinnin seurauksista.
%• Luonnehdinta artikkelin vartalon jäsennyksestä (ellei sisälly edellisiin)
%
%Johdanto sisältää välttämättä viittauksia relevanttiin kirjallisuuteen:
%keskeisen taustakirjallisuuden pitää käydä ilmi johdannosta.
%Johdantoa ei saa kuormittaa teknisellä terminologialla, käsitteiden määritelmillä, monimutkaisella matematiikalla eikä syvällisellä kirjallisuuden
%tarkastelulla; niiden paikka on muualla.
%Johdannon pitää olla artikkelin helppolukuisin kohta, eikä se saa olla
%kovin pitkä

%motivaatio: miksi ongelma on mielenkiintoinen? mitkä ovat merkittävät seikat

%todo: työn rakenne kuitenkaan toistamatta liikaa sisällysluetteloa

\section{workname: Taustaluku}
\subsection{Relaatiomalli}

Relaatiotietokanta on relaatiomalliin \cite{codd1970relational} perustuva tietokanta. Relaatiomallin keskeinen piirre on kaiken datan esittäminen n-paikkaisen karteesisen tulon osajoukkona, ja se tarjoaa deklaratiivisen menetelmän datan ja kyselyjen määrittämiseen. Relaatiomalli koostuu attribuuteista, monikoista ja relaatioista. Matemaattisessa määritelmässä attribuutti on pari joka sisältää attribuutin nimen ja tyypin sekä jokaiseen attribuuttiin liittyy sen arvojoukko. Monikko on järjestetty joukko attribuuttien arvoja. Relaatio koostuu otsakkeesta ja sisällöstä, jossa otsake on joukko attribuutteja ja keho on joukko monikkoja. Relaation otsake on myös jokaisen monikon otsake. Visuaalisessa esityksissä relaatio on taulukko ja monikko taulukon rivi. 

\begin{figure}[!h]
  \caption{Relaatiomalliin perustuva tietokanta}
  \centering
    \includegraphics[width=\textwidth]{rlt_wiki.png}
\end{figure}

\subsection{Optimoijan tavoitteet}
Tietokantakyselyjen optimoinnilla viitataan tietokantakyselyn suorittamiseen mahdollisimman tehokkaasti. Optimoinnin tavoitteena on joko maksimoida suorituskyky annetuilla resursseilla tai minimodia resurssien käyttö. Mitattavia resursseja ovat suorittimen ja muistin käyttö sekä kommunikointikustannukset. \cite{jarke1984} Muistin käyttö jakautuu tallennuskustannukseen sekä ulkomuistiin pääsyn kustannukseen. Tallennuskustannuksella tarkoitetaan ulkomuistin sekä puskurimuistin käyttöä, ja se tulee aiheelliseksi kun muistin käyttö aiheutuu pullonkaulaksi. Kommunikointikustannukset käsittävät tiedon viennin tallennuspaikasta laskentapaikkaan ja edelleen tulosten esityspaikkaan. Ne jakautuvat kommunikaatioväylän käyttökustannukseen ja tiedonsiirrosta aiheutuvaan suorittamisen viiveeseen.

Resurssin merkitys riippuu tietokantatyypistä. Hajautetuissa tietokannoissa hitailla yhteysväylillä kommunikointikustannukset hallitsevat kustannuksia. Paikallisesti hajautetuissa tietokannoissa kaikilla resursseilla on sama painoarvo. Keskitetyissä tietokannoissa ulkomuistiin pääsyn kustannus ja prosessorin käyttö ovat oleellisia. \cite{jarke1984} Tässä tutkielmassa keskitymme erityisesti keskitettyjen tietokantojen optimointiin.
\subsection{Optimoijan toiminta}
%Käy tässä käpi kyselyn optimoijan koko stäkki
Tietokannan hallintajärjestelmän suorittama SQL-kyselyn prosessointi sisältää useita vaiheita. Kuva 2 esittää kyselyn optimoijan rakenteen hallintajärjestelmän sisällä.
Kyselyn prosessointi alkaa kyselyn jäsentäjän suorittamalla kyselyn syntaksin ja semantiikan oikeellisuuden validoinnilla. \cite{oracle2009doc}
Syntaksin validoinnissa jäsentäjä tarkastaa kyselyn lauseopin oikeellisuuden. Semantiikan validoinnissa tarkastetaan objektien aitous, kyselyn yksiselitteisyys, oikeus haettavaan tietoon ja muuttujien tyypin sopivuus sarakkeiden tyyppeihin. Useat hallintajärjestelmät myös
tallentavat kyselyt validoinnin jälkeen talteen, jotta samaa kyselyä ei tarvitse jäsentää ja optimoida uudelleen. Yksi esimerkki kyselyjä tallentavasta hallintajärjestelmästä on Oracle Database, jossa tallennuspaikkaa kutsutaan nimellä Shared Pool. \cite{oracle2005doc}

Seuraavaksi jäsentäjä jakaa kyselyn lohkoihin(block) siten, että yhdessä lohkossa on täsmälleen yksi SELECT-lause, yksi FROM-lause ja korkeintaan yksi WHERE-, GROUP BY- ja HAVING-lause. \cite{ramakrishnan2003database}
Kyselyn mahdollisesti sisältämät alikyselyt muodostavat kukin oman lohkonsa. %
(Alikyselyistä enemmän kappaleessa n?)

Jokainen lohko jäsennetään puuksi, joka on kyselyn algebrallinen esitysmuoto. \cite{mahajan2012}
Puun solmu sisältää yhden operaation kyselyn suorittamiseksi, ja sillä on nolla tai useampi alisolmua joiden ulostuloa(output) käytetään sen syötteenä.
Esimerkiksi join-operaatiossa solmulla on kaksi alisolmua, joille toteutetaan join-operaatio ja sort-operaatiolla on yksi alisolmu joka järjestetään.
Lehtisolmut ovat solmuja jotka suorittavat hakuja (scan) levyltä ja palauttavat saadut tulokset. 
Puu suoritetaan lehtisolmuista juureen.

Tämän jälkeen kysely uudelleenkirjoitetaan (rewrite) valitsemalla hakumetodit (access method), liittämisjärjestyksen (join orders) ja liittämistavat (join methods) tietokannan käyttämien heuristiikkojen pohjalta. Monimutkaisille kyselyille sovelletaan myös muunnossääntöjä. \cite{mahajan2012}
Uudelleenkirjoituksen tarkoitus on helpottaa optimoijan työtä parhaan kyselysuunnitelman valinnassa.
%todo: http://docs.oracle.com/cd/A97630_01/server.920/a96520/qr.htm
%http://www.budaconsulting.com/blog/bid/79324/Query-Rewrite-What-is-MySQL-Doing-To-My-Queries

Kun kysely on uudelleenkirjoitettu, lähetetään se kyselyn optimoijalle. Kyselyn optimoija hakee systeemitaulustosta tilastotietoa kyselyyn liittyvistä relaatioista ja relaatioita vastaavista kyselysuunnitelmista. Optimoija määrittää lohkojen suoritusjärjestyksen ja prosessoi jokaisen lohkon FROM-lauseeseen liittyvät relaatiot. Mikäli lohkossa on useampi relaatio, arvioi optimoija join-operandin permutaatioiden suorituskyvyt. Optimoija rakentaa lohkoja vastaavat kyselysuunnitelmat käyttäen systeemitauluston tilastotietoa ja valitsee niistä pienimmän kustannuksen sisältävän kyselysuunnitelman. Seuraavassa vaiheessa kysely suoritetaan käyttämällä optimoijan tuottamaa kyselysuunnitelmaa. Suorittamisessa kyselysuunnitelma muutetaan suoritettavaksi konekieleksi ja kyselyn lähteen mukaan joko suoritetaan tai tallennetaan muistiin myöhempää suorittamista varten. (Miten viitataan koko kappaleeseen?) (\cite{selinger1979access})

\begin{figure}[!h]
  \caption{Kyselyn jäsentäminen, optimointi ja suoritus}
  \centering
    \includegraphics[width=\textwidth]{optimoija60.png}
\end{figure}

%\section{Kyselyn jäsentäminen} Toistaiseksi skipataan koko kappale.
%http://docs.oracle.com/cd/B28359_01/server.111/b28318/sqlplsql.htm#i1243
%Kyselyn jäsentämisessä hallintajärjestelmä pyrkii validoimaan kyselyn syntaksin sekä semantiikan oikeellisuuden. 
%Syntaksin validoinnissa jäsentäjä tarkastaa kyselyn lauseopin oikeellisuuden

\section{Kyselyn optimointi}
Kyselyn optimoijan tulee arvioida kustannus jokaiselle kyselysuunnitelmalle. Kustannusarviointi koostuu kahdesta vaiheesta: ensiksi arvioidaan jäsennetyn puun jokaisen alkion operaation suorittamiseen kuluva aika. (pipelining, temppirelaatiot.)
Tämän jälkeen arvioidaan jokaisen alkion tulosjoukon koko, sekä lisäksi tarkastetaan onko tulosjoukko järjestetty. Solmun tulosjoukko on ylisolmun syöte, joten sen koko ja järjestys vaikuttavat suoraan ylisolmun arviointiin.

\subsection{Kyselysuunnitelmien tuottaminen}
Kyselysuunnitelma koostuu laajennetusta relaatioalgebrapuusta, jossa jokainen solmu kuvaa algebrallista operaatiota. Solmuun on liitetty tieto käytettävästä hakumetodista tiedon hakemiseen taulusta ja suoritusmetodista relaatio-operaation suoritukseen. Tutkitaan seuraavaa SQL-kyselyä:

\begin{frame}

SELECT K.nimi

FROM Kirjat K, Teokset T

WHERE K.tid = T.tid AND K.kid = 5 AND T.arvosana $\textgreater$  3
\end{frame}
\\\newline
Kysely voidaan tulkita relaatioalgebrassa seuraavasti:

\begin{frame}

$\pi$$_{nimi}$(\sigma_{kid=5 \wedge arvosana \textgreater 3}(Kirjat$\Join$$ _{tid=tid}$Teokset))
\end{frame}
\\\newline
Kysely voidaan suorittaa esimerkiksi käyttäen join-operaatiossa nested loop join-algoritmia, jonka jälkeen jokaiselle riville suoritetaan 
valinta ja projektio. Kyselyä vastaava kyselysuunnitelma on siten seuraava:

\begin{figure}[!h]
  \caption{Kyselysuunnitelma erimerkkikyselylle}
  \centering
    \includegraphics[width=\textwidth]{queryevalplan60.png}
\end{figure}

\subsection{Kyselysuunnitelmien kustannusarviointi}

\subsection{Tulosjoukon koon arviointi}
Operaation kustannus riippuu syötteen koosta. Tutkitaan seuraavaa tapausta:
\newline

\begin{frame}

SELECT attribuuttit

FROM relaatiot

WHERE ehto 1 $\wedge$ ehto 2 $\wedge$ ... $\wedge$ ehto n
\end{frame}
\\\newline
Kyselyn palauttamien monikkojen maksimimäärä on relaatioiden karteesinen tulo. Jokainen WHERE-ehto harventaa monikkojen määrää. WHERE-ehdon vaikutusta tulosjoukon kokoon voidaan mallintaa lisäämällä jokaiseen ehtoon vähennyskerroin, joka on oletettu suhde lähtöjoukosta tulosjoukkoon vain kyseisen ehdon osalta. 
Tulosjoukon koko voidaan siten arvioida kertomalla maksimijoukko vähennyskertoimien tulolla. \cite{ramakrishnan2003database}

WHERE-lauseen ehtojen kertoimia voidaan laskea hyödyntämällä systeemitaulustoon tallennettua tilastotietoa. Mikäli relaatiolle ei ole tilastotietoa saatavilla, oletetaan relaation olevan pieni ja käytetään määriteltyä vakiota lasketun kertoimen sijaan. \cite{selinger1979access} Oletetaan seuraavat tiedot: %source: https://www.cs.duke.edu/courses/compsci516/cps216/spring03/papers/selinger-etal-1979.pdf
\\\newline
NKeys(I) = eri avainten lukumäärä indeksissä I
\\
F = vähennyskerroin
\\\newline
$sarake = arvo$ 
\\
tyyppiselle ehdolle vähennyskerroin voidaan arvioida kaavalla $F = \frac{1}{NKeys(I)}$, jos sarakkeessa on indeksi kyseiselle relaatiolle. \cite{selinger1979access} Ilman indeksiä kyselyn optimoija käyttää kiinteää arvoa vähennyskertoimen arvioimiseen, joka esimerkiksi System R-relaatiotietokantaohjelmassa on 1/10.
\\\newline
$sarake1 = sarake2$ 
\\
tapauksessa voidaan vähennyskerroin arvioida kaavalla $F = \frac{1}{MAX (NKeys(I1), NKeys(I2))}$ jos kummassakin sarakkeessa on indeksi. Lisäksi oletetaan, että jokaisesta pienemmän indeksin arvoa vastaa arvo toisesta indeksistä. Mikäli vain toisessa sarakkeessa on indeksi, 
voidaan kustannus laskea aiemmalla kaavalla käyttäen indeksin omaavaa saraketta. Mikäli kummassakaan sarakkeessa ei ole indeksiä, arvioidaan arvoksi 1/10.
\\\newline
$sarake > arvo$
\\
tapauksessa voidaan käyttää kaavaa $F = \frac{\textit{suurin avain - arvo}}{\textit{suurin avain - pienin avain}}$. Mikäli sarake ei ole aritmeettinen tai arvoa ei tiedetä käytetään vakiona arvoa 1/3.
\\\newline
\textit{sarake IN (lista arvoja)}
\\
$F = koko(lista) \times (sarake = arvo)$

\subsection{Systeemitaulustoon tallennettu tilastotieto}
Kustannusten arviointi vaatii tilastotietoa. \cite{ramakrishnan2003database}

%Eka osa tähän
todo: täsmennä ja siisti
(sis. systeemitaulustoon tallennetun tilastotiedon käytön)
Tilastotieto on..
Tilastotiedoilla lasketaan kyselysuunnitelmien kustannusarvio ja valitaan pienimmän kustannusarvion omaava suunnitelma.
Tietoa ei tallenneta jatkuvasti, sillä se aiheuttaisi esimerkiksi rinnakkaisuusongelmia. Tietokannan valvoja 
triggeriöi päivityksen esim. komennolla optimize db. Tilastotiedon harvan päivityksen takia valituksi ei aina välttämättä tule 
tehokkain suunnitelma.

%mielenkiintoisia aiheita:
%NESTED SUBQUERIES
%\section{Fyysinen optimointi (physical optimization)}
\section{Alikyselyt}

Kyselyn predikaatin operandin ollessa toinen kysely puhutaan alikyselystä. Alikyselyt suoritetaan..

Alikysely voi palauttaa yhden arvon tai listan arvoja riippuen sen operandista. Tutkitaan seuraavaa kyselyä:

\newline

\begin{frame}

SELECT nimi

FROM Kirjat

WHERE hinta = (SELECT AVG(hinta)
		
	FROM Kirjat)
\end{frame}
\\\newline
Esimerkkikyselyssä alikysely palauttaa yhden arvon, AVG(hinta). Tämä arvo lisätään ylikyselyyn predikaatiksi niinkuin se olisi ollut osa ylempää kyselyä. Esimerkkitapauksessa keskiarvon ollessa 10 predikaatti on "hinta = 10".

Mikäli alikysely palauttaa useamman arvon, kuten seuraavassa tapauksessa:

\newline

\begin{frame}

SELECT nimi

FROM Kirjat

WHERE hinta IN (SELECT hinta
		
	FROM Kirjat

	WHERE genre = 'scifi')
\end{frame}
\\\newline

käytetään alikyselyn palauttamaa listaa ylikyselyn predikaattina. Esimerkkikyselyn palauttaessa listan (5,15) vastaava ylikyselin predikaatti on "hinta IN (5,15).


\section{Optimoijan käyttäminen (hints)}
\section{Käyttöesimerkki, suorituskykyvertailu}
\section{yhteenveto}

% Write some science here.

%CITES
%\cite{jarke1984}
%\cite{chaudhuri1998}
%\cite{freytag1987rule}
%\cite{ono1990}
%\cite{graefe1993query}
%\cite{chu1999least}
%\cite{aho1979}
%\cite{cole1994}
%\cite{mahajan2012}

% --- Back matter ---
%
% bibtex is used to generate the bibliography. The babplain style
% will generate numeric references (e.g. [1]) appropriate for theoretical
% computer science. If you need alphanumeric references (e.g [Tur90]), use
%
% 
%
% instead.

%\bibliographystyle{unsrt}
\bibliographystyle{babalpha}
\bibliography{references-fi}


\end{document}
