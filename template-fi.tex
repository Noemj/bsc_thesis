% --- Template for thesis / report with tktltiki2 class ---

\documentclass[finnish]{tktltiki2}

% tktltiki2 automatically loads babel, so you can simply
% give the language parameter (e.g. finnish, swedish, english, british) as
% a parameter for the class: \documentclass[finnish]{tktltiki2}.
% The information on title and abstract is generated automatically depending on
% the language, see below if you need to change any of these manually.
% 
% Class options:
% - grading                 -- Print labels for grading information on the front page.
% - disablelastpagecounter  -- Disables the automatic generation of page number information
%                              in the abstract. See also \numberofpagesinformation{} command below.
%
% The class also respects the following options of article class:
%   10pt, 11pt, 12pt, final, draft, oneside, twoside,
%   openright, openany, onecolumn, twocolumn, leqno, fleqn
%
% The default font size is 11pt. The paper size used is A4, other sizes are not supported.
%
% rubber: module pdftex

% --- General packages ---

\usepackage[utf8]{inputenc}
\usepackage{lmodern}
\usepackage{microtype}
\usepackage{amsfonts,amsmath,amssymb,amsthm,booktabs,color,enumitem,graphicx}
\usepackage[pdftex,hidelinks]{hyperref}

% Automatically set the PDF metadata fields
\makeatletter
\AtBeginDocument{\hypersetup{pdftitle = {\@title}, pdfauthor = {\@author}}}
\makeatother

% --- Language-related settings ---
%
% these should be modified according to your language

% babelbib for non-english bibliography using bibtex
\usepackage[fixlanguage]{babelbib}
\selectbiblanguage{finnish}

% add bibliography to the table of contents
\usepackage[nottoc]{tocbibind}
% tocbibind renames the bibliography, use the following to change it back
\settocbibname{Lähteet}

% --- Theorem environment definitions ---

\newtheorem{lau}{Lause}
\newtheorem{lem}[lau]{Lemma}
\newtheorem{kor}[lau]{Korollaari}

\theoremstyle{definition}
\newtheorem{maar}[lau]{Määritelmä}
\newtheorem{ong}{Ongelma}
\newtheorem{alg}[lau]{Algoritmi}
\newtheorem{esim}[lau]{Esimerkki}

\theoremstyle{remark}
\newtheorem*{huom}{Huomautus}


% --- tktltiki2 options ---
%
% The following commands define the information used to generate title and
% abstract pages. The following entries should be always specified:

\title{Tietokantakyselyjen optimointi relaatiotietokannassa}
\author{Olli Rissanen}
\date{\today}
\level{Kandidaatintutkielma}
\abstract{Tutkielmassa tutustutaan tietokantakyselyjen optimointiin relaatiotietokantojen hallintajärjestelmien osalta sekä optimoinnin vaikutukseen kyselyjen suorituskyvyssä. }

% The following can be used to specify keywords and classification of the paper:

\keywords{Information systems → Query optimization}
\classification{} % classification according to ACM Computing Classification System (http://www.acm.org/about/class/)
                  % This is probably mostly relevant for computer scientists

% If the automatic page number counting is not working as desired in your case,
% uncomment the following to manually set the number of pages displayed in the abstract page:
%
% \numberofpagesinformation{16 sivua + 10 sivua liitteissä}

\begin{document}

% --- Front matter ---

\maketitle
\makeabstract
\tableofcontents
\newpage


% --- Main matter ---

\section{Johdanto}
%Konteksti
Tietokantojen suorituskyky on yhä tärkeämpää tiedon määrän kasvaessa. Optimoimalla tietokantakyselyjen suoritusta voidaan helpottaa käyttäjien tiedonhakua sekä kasvattaa tietokannan suorituskykyä.\cite{mor2012} Erityisesti olio-relaatiokuvausta (ORM, object-relational mapping) toteuttavissa ohjelmointikielissä puhtaan SQL-kyselyjen kirjoittaminen on siirretty käyttäjältä alemmalle tasolle. ORM luo relaatiotietokannan pohjalta käyttäjälle oliotietokannan, jonka suorituskyky on kuitenkin sidottu relaatiotietokantaan. Relaatiotietokannan kyselyjen optimoinnilla pystytään siten saavuttamaan .. [viite ois kiva]

Tietokannan hallintajärjestelmä on kokoelma ohjelmia tiedon tallentamiseen, muokkaamiseen, analysointiin ja keräämiseen tietokannasta. Hallintajärjestelmää käytetään kyselykielillä, joista esimerkiksi SQL on suunniteltu relaatiotietokantojen hallintajärjestelmille. Kyselyn optimointi on toteutettu automaattisena toimenpiteenä tietokannan hallintojärjestelmän sisältämässä kyselyoptimoijassa, ja se on ydintekijä erityisesti relaatiomalliin pohjautuvien hallintajärjestelmien menestyksessä. Kyselyoptimoijan tavoitteena on minimoida itse optimointiin käytetty aika ja maksimoida optimoinnista saatu hyöty.\cite{jarke1984} 

Kyselyoptimoija toimii etsien kyselyä vastaavat mahdolliset kyselysuunnitelmat ja valitsemalla niistä tehokkaimman. Kyselysuunnitelma sisältää sarjan algebrallisia operaatioita tietokannan relaatioille jotka tuottavat tulokseksi halutun vastauksen. Tietokantakyselyä vastaavia kyselysuunnitelmia voi olla useita, sillä kyselyiden algebralliset esitykset voidaan usein esittää monena loogisesti vastaavana esityksenä. Algebrallista operaatiota kohden voi myös löytyä useita toteutuksia, kuten join-operaatiota toteuttavat merge join ja hash join. Toisiaan vastaavat esitykset voivat olla suorituskyvyltään jopa eri asteikolla.
%Ongelma

Haasteeksi nousee kyselysuunnitelman luominen ja kyselysuunnitelmien suorituskyvyn ennustaminen. Optimointi on vaikea hakuongelma, jossa hakualue voi nousta erittäin suureksi kyselyn ollessa monimutkainen.\cite{chaudhuri1998} Optimoijan tulee valita pienin mahdollinen hakualue, joka pitää sisällään halvimmat suunnitelmat. Suorituskyvyn ennustamisen ja hakualueen rajauksen lisäksi optimoija tarvitsee tehokkaan algoritmin koko hakualueen läpikäymiseen.

Tutkielman rakenteesta

%
%• Yleinen selostus tutkimusaiheen laajemmasta tutkimusalueesta, tutkimuksen kontekstista.
%• Ratkaistavan tutkimusongelman selostus (vaikeus, esteet, haasteet).
%• Lyhyt katsaus ongelman olemassa oleviin tai tavanomaisiin ratkaisuihin ja niiden rajoitteisiin.
%• Hahmotelma tai yhteenveto ehdotettavasta uudesta ratkaisusta.
%• Yhteenveto ratkaisun arvioimistavasta ja arvioinnin seurauksista.
%• Luonnehdinta artikkelin vartalon jäsennyksestä (ellei sisälly edellisiin)
%
%Johdanto sisältää välttämättä viittauksia relevanttiin kirjallisuuteen:
%keskeisen taustakirjallisuuden pitää käydä ilmi johdannosta.
%Johdantoa ei saa kuormittaa teknisellä terminologialla, käsitteiden määritelmillä, monimutkaisella matematiikalla eikä syvällisellä kirjallisuuden
%tarkastelulla; niiden paikka on muualla.
%Johdannon pitää olla artikkelin helppolukuisin kohta, eikä se saa olla
%kovin pitkä

%motivaatio: miksi ongelma on mielenkiintoinen? mitkä ovat merkittävät seikat

%todo: työn rakenne kuitenkaan toistamatta liikaa sisällysluetteloa

\section{workname: Taustaluku}
\subsection{Relaatiomalli}

Relaatiotietokanta on relaatiomalliin perustuva tietokanta. Relaatiomallin keskeinen piirre on kaiken datan esittäminen n-paikkaisen karteesisen tulon osajoukkona, ja se tarjoaa deklaratiivisen menetelmän datan ja kyselyjen määrittämiseen. Relaatiomalli koostuu attribuuteista, monikoista ja relaatioista. Matemaattisessa määritelmässä attribuutti on pari joka sisältää attribuutin nimen ja tyypin sekä jokaiseen attribuuttiin liittyy sen arvojoukko. Monikko on järjestetty joukko attribuuttien arvoja. Relaatio koostuu otsakkeesta ja sisällöstä(body?), jossa otsake on joukko attribuutteja ja keho on joukko monikkoja. Relaation otsake on myös jokaisen monikon otsake. Visuaalisessa esityksissä relaatio on taulukko ja monikko taulukon rivi. 

\begin{figure}[h!]
  \caption{Relaatiomalliin perustuva tietokanta}
  \centering
    \includegraphics[width=\textwidth]{rlt_wiki.png}
\end{figure}

\subsection{Kyselyoptimoijan toiminta}
Tietokantakyselyiden optimoinnilla viitataan tietokantakyselyn suorittamiseen mahdollisimman tehokkaasti. Optimoinnin tavoitteena on joko maksimoida suorituskyky annetuilla resursseilla tai minimodia resurssien käyttö. Mitattavia resursseja ovat suorittimen ja muistin käyttö sekä kommunikointikustannukset. Muistin käyttö jakautuu tallennuskustannukseen sekä ulkomuistiin pääsyn kustannukseen. Tallennuskustannuksella tarkoitetaan ulkomuistin sekä puskurimuistin käyttöä, ja se tulee aiheelliseksi kun muistin käyttö aiheutuu pullonkaulaksi. 

Resurssin merkitys riippuu tietokantatyypistä. Hajautetuissa tietokannoissa hitailla yhteysväylillä kommunikointikustannukset hallitsevat kustannuksia. Paikallisesti hajautetuissa tietokannoissa kaikilla resursseilla on sama painoarvo. Keskitetyissä tietokannoissa ulkomuistiin pääsyn kustannus ja prosessorin käyttö ovat oleellisia. Tämän tutkielman aihepiiriin kuuluu vain keskitettyjen tietokantojen optimointi.

Tietokannan hallintajärjestelmän suorittama SQL-kyselyn prosessointi sisältää useita vaiheita. Aluksi hallintajärjestelmä jäsentää kyselyn syntaktisen ja semanttisen tarkastelun mahdollistamiseksi. Tämän jälkeen kysely muutetaan jäsennetyksi puuksi, joka on kyselyn algebrallinen esitysmuoto. \cite{mahajan2012}

todo:validoi ja siisti: Kysely uudelleenkirjoitetaan (rewrite) tämän jälkeen valitsemalla hakumetodit (access method), liittämisjärjestyksen (join orders) ja liittämistavat (join methods) tietokannan käyttämien heuristiikkojen pohjalta. Monimutkaisille kyselyille sovelletaan myös muunnossääntöjä. \cite{mahajan2012}

Seuraava vaihe on kyselyn optimointi. Kyselyoptimoijan sisääntulo(input) sisältää kyselyn, tietokannan mallin sisältäen taulujen ja indeksien määritelmät sekä tietokannan tilastoja. Kyselystä haetaan predikaatit, indeksit ja liitokset (joins) kyselysuunnitelmaa varten. Tämän jälkeen luodaan kyselyä vastaavat kyselysuunnitelmat ja arvioidaan niistä paras. Seuraavassa vaiheessa kysely suoritetaan käyttämällä optimoijan tuottamaa kyselysuunnitelmaa.

todo: SQL ja relaatiomalli

todo: relaatiotietokanta vs no-sql

todo: optimoi menee

%CITES
\cite{jarke1984}
\cite{chaudhuri1998}
\cite{freytag1987rule}
\cite{ono1990}
\cite{graefe1993query}
\cite{chu1999least}
\cite{aho1979}
\cite{cole1994}
\cite{mahajan2012}

\section{Tietohakemistoon tallennettu tilastotieto}
todo: täsmennä ja siisti

Tilastotiedoilla lasketaan kyselysuunnitelmien kustannusarvio ja valitaan pienimmän kustannusarvion omaava suunnitelma.
Tietoa ei tallenneta jatkuvasti, sillä se aiheuttaisi esimerkiksi rinnakkaisuusongelmia. Tietokannan valvoja 
triggeriöi päivityksen esim. komennolla optimize db. Tilastotiedon harvan päivityksen takia valituksi ei aina välttämättä tule 
tehokkain suunnitelma.
\section{Algebralliset muunnoslait} %kandi4
\section{menetelmä n}
\section{menetelmien vertailu}
\section{case study?}
\section{yhteenveto}

% Write some science here.


% --- Back matter ---
%
% bibtex is used to generate the bibliography. The babplain style
% will generate numeric references (e.g. [1]) appropriate for theoretical
% computer science. If you need alphanumeric references (e.g [Tur90]), use
%
% \bibliographystyle{babalpha}
%
% instead.

\bibliographystyle{unsrt}
\bibliography{references-fi}


\end{document}
